\documentclass[fleqn,10pt]{wlscirep}
\usepackage[utf8]{inputenc}
\usepackage[T1]{fontenc}
\usepackage{bm}
\usepackage{caption}
\usepackage{subcaption}
\usepackage{newtxmath} % MER: Prettier math
%\usepackage{natbib}
\usepackage{cleveref} % MER: Allows \Cref (but use Section X.Y not Subsection X.Y)
\usepackage{booktabs} % MER: Prettier tables

\newcommand{\R}{\mathbb{R}} 

\title{Constructing a benchmark collection for simulating ionic electrodiffusion in cellular geometries}
\author[1]{Pietro Benedusi, Marius Causemann, Ingvild S.~Devold, Jørgen S.~Dokken, Ada J.~Ellingsrud, Halvor Herlyng, Marie E. Rognes}
\affil[1]{Department of Numerical Analysis and Scientific Computing, Simula Research Laboratory, Oslo, Norway}
\affil[*]{meg@simula.no}

\begin{abstract}
\end{abstract}
\begin{document}

\flushbottom
\maketitle
% * <john.hammersley@gmail.com> 2015-02-09T12:07:31.197Z:
%
%  Click the title above to edit the author information and abstract
%
\thispagestyle{empty}

\section{The KNP-EMI equations}

This presentation is directly based on the problem formulation previously defined~\cite{ellingsrud2020finite, benedusi2024scalable}.

We consider a domain $\Omega = \Omega_i \cup \Omega_e \cup \Gamma \subset \R^d$ for $d \in \{2,3\}$ with $\Omega_i$ and $\Omega_e$ representing the intra- and extracellular domains (ICS, ECS), respectively, and $\Gamma = \partial\Omega_i$ the cellular membrane(s). In general, the intracellular domain can be composed of $N_{\rm cell}\in \mathbb{N}$ disjoint domains $\Omega_i = \bigcup_{j=1}^{N_{\rm cell}}\Omega_{i,j}$, with $\Omega_{i,j}$ representing the $j$th (biological) cell with membrane $\Gamma_j=\partial\Omega_{i,j}$. We consider a set $K$ of ionic species, for example $K = \{\text{Na}^+, \text{K}^+, \text{Cl}^-\}$,  with cardinality $|K|$. For each ionic species $k \in K$ and physical region $r\in\{i,e\}$, we model the evolution of ionic concentrations $[k]_r: \overline{\Omega}_r \times [0,T] \to \R$, as well as electric potentials $\phi_r:\overline{\Omega}_r \times[0,T] \to \R$. We also define the membrane potential $\phi_M : \Gamma \times (0,T] \to \R$ as the jump in the electric potential:
\begin{equation}
  \phi_M(x,t) = \phi_i(x,t) - \phi_e(x,t)  \quad \text{for } x \in \Gamma,\, t > 0.
   \label{eq:membrane-potential}
\end{equation}

\subsection{Equations in the intracellular and extracellular domains}

For each region $r \in \{i, e\}$ and for each ionic species $k \in K$ and for $t\in (0,T)$ and $x\in\Omega_r$, we consider the $2 \times (|K| + 1)$ time-dependent, partial differential equations
\begin{subequations}
  \begin{align}
    \partial_t [k]_r(x,t) + \nabla \cdot J_r^k(x,t) &= f_r^k(x,t), 
    \label{eq:conservation-ions} \\
    \sum_{k\in K}z_k \nabla \cdot J_r^k(x,t) &= 0 .
    \label{eq:knp}
  \end{align}
\end{subequations}
Here $J_r^k:\overline{\Omega}_r\times[0,T]\to\R^d$ is the ion flux density. We consider a Nernst--Planck relation describing diffusion and drift in the electric field induced by the potential $\phi_r$:
\begin{equation}
  J_r^k = J_{r,\mathrm{diff}}^k + J_{r,\mathrm{drift}}^k  
   = - D_r^k \nabla [k]_r - \frac{D_r^k z_k}{\psi} [k]_r \nabla \phi_r,
  \label{eq:flux}
\end{equation}
where $D_r^k\in\R_+$ is the effective diffusion coefficient of ionic species $k$ in region $r$, $z_k$ is the valence of ion $k$, and $\psi = R T F^{-1}$ is composed of the gas constant $R$, the absolute temperature $T$, and the Faraday constant $F$. Finally, $f_r^k:\Omega_r\times[0,T]\to\R$ represent bulk source terms. 

\subsection{Interface conditions}

We assume that the total ionic current density $I_M:\Gamma\times(0,T]\to\R$ is continuous across $\Gamma$ such that for $t > 0$ and $x \in \Gamma$,
\begin{equation}\label{eq::I_M1}
  I_M(x,t) = F \sum_{k \in K} z_k J_i^k(x,t) \cdot n_i(x) = - F \sum_{k \in K} z_k J_e^k(x,t) \cdot n_e(x),
\end{equation}
where $n_r$ denotes the normal on the interface pointing out from $\Omega_r$. We further assume that $I_M$ consists of two components: a total ionic channel current $I_{\rm ch}$ and a capacitive current $I_{\rm cap}$. Both of these currents have contributions from each ionic species: 
\begin{equation}\label{eq::I_M2}
I_M = I_{\rm ch} + I_{\rm cap} = \sum_{k \in K} I_{\rm ch}^k + \sum_{k \in K} I_{\rm cap}^k .
\end{equation}
The ionic channel currents $I_{\rm ch}^k$ will be subject to modelling, while we consider the capacitor equation for the total capacitive current
\begin{equation}
I_{\rm cap} = C_m \partial_t \phi_M ,
\end{equation}
where $C_m\in\R_+$ is the membrane capacitance. 

As in~\cite[Section 2.1.3]{ellingsrud2020finite}, we derive an expression for $J_r^k \cdot n_r$ that will be of useful in the weak formulation of the KNP-EMI equations. For $k \in K$ and $r \in \{i, e\}$, we introduce the relation
\begin{equation}
  \label{eq::I_cap}
I_{\rm cap}^k = \alpha_r^kI_{\rm cap} = \alpha_r^k C_m \partial_t \phi_M,
\end{equation}
and define the ratio
\begin{equation}
  \alpha_r^k = \frac{D_r^kz_k^2[k]_r}{\sum_{\ell \in K}D_r^\ell z_\ell^2[\ell]_r} \in [0,1],
  \quad\text{with} \quad \sum_{k \in K} \alpha_r^k=1.
\end{equation}
Combining~\eqref{eq::I_M1}, \eqref{eq::I_M2} and \eqref{eq::I_cap}, we can express the normal fluxes corresponding to a specific ion $k\in K$, as a function of the ionic currents and the membrane potential:
\begin{equation}\label{eq::J_n}
J_r^k\cdot n_r= \pm_r\frac{I_{\rm ch}^k + I_{\rm cap}^k}{Fz_k}=\pm_r\frac{I_{\rm ch}^k + \alpha_r^kC_m \partial_t \phi_M}{Fz_k},
\end{equation}
for $r \in \{i, e\}$, $\pm_i=1$, and $\pm_e=-1$.

\subsection{Initial conditions}
\label{sec::ic}
To close the KNP-EMI system, we impose initial conditions for $k\in K$ and $r\in\{i,e\}$:
%
\begin{align}
&[k]_r(x,0)=[k]_r^0(x), &&\text{for } x\in\Omega_r,\\ 
&\phi_M(x,0)=\phi_M^0(x), &&\text{for } x\in\Gamma,
\end{align}
with $[k]_r^0$ and $\phi_M^0$ representing initial ion concentration and an initial membrane potential, respectively.

\subsection{Boundary conditions}
\label{sec::bc}
To close the KNP-EMI system, we impose initial conditions for $k\in K$ and $r\in\{i,e\}$:
%
\begin{align}
&[k]_r(x,0)=[k]_r^0(x), &&\text{for } x\in\Omega_r,\\ 
&\phi_M(x,0)=\phi_M^0(x), &&\text{for } x\in\Gamma,
\end{align}
with $[k]_r^0$ and $\phi_M^0$ representing initial ion concentration and an initial membrane potential, respectively.

We impose homogeneous Neumann boundary conditions on the exterior boundary $\partial \Omega$
\begin{equation}\label{eq::bc}
  J_r^k(x,t) \cdot n_r(x) = 0 \quad \text{for } x \in \partial\Omega.
\end{equation}
%i.e. a no-flux condition for concentrations and potentials, implying that the total number of ions is conserved in $\Omega$. 
% Considering a Dirichlet condition would not alter the essence of this work.
% With homogeneous Neumann boundary conditions on the whole of $\partial\Omega$, 
The electric potentials $\phi_r$ in \eqref{eq:flux} are only determined up to a (common) additive constant. Therefore, an additional constraint must be considered to enforce uniqueness. Previous works have for example considered a zero-mean condition on $\phi_e$~\cite{ellingsrud2020finite}. 

\subsection{Ionic membrane models}
\label{sec::ion}


\newpage
\section{Benchmark examples}

The description of Model A is directly based on our previous manuscript~\cite{benedusi2024scalable}.

\subsection{Model A (2D)}

We consider an idealized geometry for a single cell represented by a $2$-dimensional square $\Omega=[0,1]^2 \times 10^{-6}$ $\mu$m, with $\Omega_i=[0.25,0.75]^d$ $\mu$m. We consider the active Hodgkin-Huxley membrane model and as stimulus impose a time-periodic sodium current on the entire membrane $\Gamma$:
\begin{equation}
  {g}^{\rm Na}_{\rm stim}= 40 e^{-(t\,\text{mod}\,\tau )/0.002},
\end{equation}
with $\tau = 10$ ms the time interval between subsequent stimuli. The
other parameters are as listed in
Table~\ref{tab:parameters:modelA}. Relevant quantities of interest are
the evolution of concentrations and potential over time until $T=0.3$s.
\begin{table}
  \begin{center}
    \begin{tabular}{lllll}
      \toprule
      Parameter & Symbol & Value & Unit & Ref. \\
      \midrule
      gas constant       & $R$ &  8.314             & J/(K mol)   &  \\
      temperature        & $T$ &  300               & K           & \\
      Faraday constant & $F$ &  $9.648\cdot 10^4$ & C/mol       & \\
      membrane capacitance & $C_M$ & $0.02$ & F& \\
      \midrule
      Na\textsuperscript{+} diffusion coefficient & $D^\text{Na}_r$ & $1.33\cdot10^{-9}$ & m\textsuperscript{2}/s & \cite{hille2001ion}\\
      K\textsuperscript{+} diffusion coefficient  & $D^\text{K}_r$  & $1.96\cdot10^{-9}$ & m\textsuperscript{2}/s & \cite{hille2001ion}\\
      Cl\textsuperscript{\textminus} diffusion coefficient & $D^\text{Cl}_r$ &  $2.03\cdot10^{-9}$ & m\textsuperscript{2}/s & \cite{hille2001ion}\\
      \midrule
      initial ICS Na\textsuperscript{+} concentration & $[\text{Na}^+]_i^0$ & 12 & mM & \cite{pods2013electrodiffusion} \\
      initial ECS Na\textsuperscript{+} concentration & $[\text{Na}^+]_e^0$ & 100& mM & \cite{pods2013electrodiffusion} \\
      initial ICS K\textsuperscript{+} concentration  & $[\text{K}^+]_i^0$ & 125 & mM & \cite{pods2013electrodiffusion}\\
      initial ECS K\textsuperscript{+} concentration &$[\text{K}^+]_e^0$ & 4 & mM & \cite{pods2013electrodiffusion}\\
      initial ICS Cl\textsuperscript{\textminus} concentration & $[\text{Cl}^-]_i^0$ & 137 & mM & \cite{pods2013electrodiffusion}\\
      initial ECS Cl\textsuperscript{\textminus} concentration & $[\text{Cl}^-]_e^0$ & 104 & mM & \cite{pods2013electrodiffusion}\\
      initial membrane potential & $\phi_M^0$ & $-67.74 \times 10^{-3}$ & V &  \\
      \midrule
      Na\textsuperscript{+} leak conductivity & $g^\text{Na}_\text{leak}$ & 1 & S/m\textsuperscript{2}  & \\
      K\textsuperscript{+} leak conductivity & $g^\text{K}_\text{leak}$  & 4 & S/m\textsuperscript{2} & \\
      Cl\textsuperscript{\textminus} leak conductivity & $g^\text{Cl}_\text{leak}$ & 0   & S/m\textsuperscript{2} & \\
      K\textsuperscript{+}  HH max conductivity & $\bar{g}^\text{K}$ &360  & S/m\textsuperscript{2} & \cite{hodgkin1952quantitative} \\
      Na\textsuperscript{+} HH max conductivity & $\bar{g}^\text{Na}$   & 1200 & S/m\textsuperscript{2} & \cite{hodgkin1952quantitative} \\
      \midrule
      resting membrane potential & $\phi_{\text{rest}}$ & $-65 \times 10^{-3}$ & V  &  \\
      initial Na\textsuperscript{+} activation  & $m_0$ & 0.0379 &  & \cite{hodgkin1952quantitative}\\
      initial Na\textsuperscript{+} inactivation & $h_0$ & 0.688 &  & \cite{hodgkin1952quantitative}\\
      initial K\textsuperscript{+} activation & $n_0$ & 0.276 &  & \cite{hodgkin1952quantitative}\\
      %maximum pump rate & $\rho_{\mathrm{pump}}$ &  $1.115\cdot 10^{-6}$ & mol/m$^2$ s & \cite{halnes2013electrodiffusive}\\
      %ICS Na\textsuperscript{+} threshold for Na/K-pump & $P_{\mathrm{Na}}$ &  10 & mol/m$^3$  & \cite{halnes2013electrodiffusive}\\
      %ECS K\textsuperscript{+} threshold for Na/K-pump & $P_{\mathrm{K}}$ &  1.5 & mol/m$^3$  & \cite{halnes2013electrodiffusive}
      % \bottomrule
    \end{tabular}
    \caption{Physical parameters and initial values for Model A with the Hodgkin-Huxley membrane model, copied from~\cite{ellingsrud2020finite, benedusi2024scalable}. Note that mM is equivalent to mol/m$^3$ (both SI units for concentration).}
    \label{tab:parameters:modelA} 
  \end{center}
\end{table}


\newpage
\bibliography{references.bib}
\end{document}
